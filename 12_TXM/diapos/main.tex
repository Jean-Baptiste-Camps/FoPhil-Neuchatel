\documentclass{beamer}
%\usetheme{Boadilla}
\usetheme{PaloAlto}
%\usetheme{Berlin}
%\usecolor{}

\usepackage{minted}

\usepackage{fontspec}
\setmainfont{Linux Libertine O}
\usepackage[english,french]{babel}

\usepackage{hyperref}

\institute{Univ. de Neuchâtel}

\title{Exploiter les données}
\subtitle{Introduction à la textométrie avec \textsc{txm}}
\author{Jean-Baptiste Camps \& Simon Gabay}
\date[FoPhil -- 14 févr. 2018]{Formation en philologie numérique:\\ encoder, exploiter, diffuser\\
12-16 février 2018}

\makeatletter 
         
        \AtBeginSection[]{% 
        \begin{frame}{Plan}%
        \small
        \tableofcontents[currentsection]%
        \end{frame} }

        %\AtBeginSubsection[]{% 
	%\begin{frame}{Plan}%
	%\small
	%\tableofcontents[currentsection,currentsubsection]%
%\end{frame} }


\makeatother 
    
    
\begin{document}

\maketitle
%\frontmatter 
  

%\mainmatter 

\begin{frame}{Objectifs}
	
	\begin{itemize}
		\item S'initier à la \alert{textométrie},
		\begin{itemize}
			\item créer des corpus;
			\item les interroger;
			\item faire (un peu) d'analyse quantitative.
		\end{itemize}
		\item dans le cadre d'un logiciel ``tout en un'' et convivial, 
	\alert{\textsc{txm}} \cite{Heiden2010},
		\item sur des cas tirés de la littérature du XVII\ieme{} siècle.
	\end{itemize}
	
\end{frame}


\begin{frame}[fragile]
\frametitle{\textsc{txm}: un logiciel de textométrie}


\begin{columns}
	\begin{column}{0.30\textwidth}
	\begin{center}
	\includegraphics[width= 0.8\textwidth]{img/txm.png}
	\end{center}
\begin{itemize}
	\item Logiciel libre et multiplateforme;
	\item développé à l'ÉNS-LSH de Lyon;
	\item \url{http://textometrie.ens-lyon.fr/}.
	\item fait tourner notamment la \textit{Base de français médiéval}
\end{itemize}
	\end{column}
	\begin{column}{0.66\textwidth}
		\begin{itemize}
			\item dévoué à la textométrie;
			\item repose sur des technologies de référence:
				\begin{itemize}
					\item \textsc{xml/tei} pour les données;
					\item \textsc{r} pour l'analyse statistique;
					\item \textsc{cqp} pour l'interrogation de corpus;
					\item \texttt{TreeTagger} pour l'annotation.
				\end{itemize}
		\end{itemize}
	\end{column}
\end{columns}
\end{frame}


\begin{frame} 
  \frametitle{Plan} 
  \tableofcontents
\end{frame}

\section{Mise en jambe: \textit{Andromaque}}

\begin{frame}{Création d'un corpus à partir de notre édition d'\textit{Andromaque}}

\begin{enumerate}
	\item Fichier, importer, import XML/W + CSV;
	\item sélectionner le dossier avec les sources et remplir les paramètres du corpus;
	\item lancer la création du corpus.
\end{enumerate}

\end{frame}

\begin{frame}{Premières fonctionnalités}
\begin{enumerate}
\item Consulter la description du corpus;
\item parcourir l'édition;
\item regarder le lexique;
\item ouvrir l'index, y chercher les occurrences de 'Seigneur';
	\begin{enumerate}
		\item clic-droit, envoyer vers les concordances;
		\item double-clic sur une occurrence pour aller au texte;
		\item clic-droit, envoyer vers les cooccurrents;
			\begin{enumerate}
				\item aller d'un cooccurrent aux concordances, puis au texte
			\end{enumerate}
		\item clic-droit, envoyer vers la progression;
	\end{enumerate}
\end{enumerate}

\end{frame}

\begin{frame}{Partitions et quelques éléments descriptifs}
\begin{enumerate}
\item créer une partition, en sélectionnant la structure \texttt{sp} et l'attribut \texttt{\@who};
\item consulter les dimensions;
\item créer une table lexicale, expérimenter avec les tris, la fusion ou suppression des colonnes, etc.
\end{enumerate}
\end{frame}

\begin{frame}{Statistiques de base}
\begin{enumerate}
\item créer une partition, en sélectionnant la structure \texttt{sp} et l'attribut \texttt{\@who};
\item consulter les dimensions;
\item créer une table lexicale, expérimenter avec les tris, la fusion des colonnes, etc.
\end{enumerate}
\end{frame}



\section{Importer des données et créer un corpus}

\section{Interroger les données}

\section{Quelques notions d'analyse quantitative}


\begin{frame}[fragile]
\frametitle{Approches principales: la forme (linguistique) ou le fond (sémantique)?}

Sans prétention à l'exhaustivité, on peut distinguer deux approches principales qui font emploi des méthodes que nous allons présenter:
\begin{itemize}
	\item \textbf{approche stylométrique}, qui se préoccupe d'attribution, datation, localisation des textes; pour ce type d'approche, on va s'intéresser à l'information graphique (variantes d'orthographe), flexionnelle (terminaisons verbales, …), etc. On va aussi tendre à s'intéresser seulement aux \alert{mots les plus fréquents} (mots-outils, mots-vides), les moins sensibles aux variations intentionnelles de leurs auteurs (genre, sujet, etc.). Le \textbf{sens des textes nous indiffère} (ou presque).
	\item \textbf{approche sémantique}, lexicométrique, etc., qui est à peu près l'inverse de la précédente: intérêt pour les mots en tant que lemmes et leurs rapports entre eux (cooccurrences), pour les réseaux de sens, les thèmes des textes, etc.
\end{itemize}
Très schématique, mais nous en reparlerons plus tard.
\end{frame}


\begin{frame}{test}

\end{frame}

\appendix

\begin{frame}[fragile]
\frametitle{Bibliographie} 

\begin{thebibliography}{Heiden et al., 2010}
	\bibitem[Heiden et al., 2010]{Heiden2010} Heiden, S., Magué, J-P., et Pincemin, B., \og{}TXM: Une plateforme logicielle open-source pour la textométrie – conception et développement\fg{}, dans \textit{Proc. of 10th International Conference on the Statistical Analysis of Textual Data - JADT 2010}, éd. Sergio Bolasco, Isabella Chiari, Luca Giuliano, Rome, 2010, t.~2, p. 1021-1032, \url{https://halshs.archives-ouvertes.fr/halshs-00549779/fr/}.
\end{thebibliography}


\end{frame}


\end{document}
