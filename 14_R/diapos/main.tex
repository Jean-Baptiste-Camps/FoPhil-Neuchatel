% !TeX spellcheck = fr
% !TeX encoding = UTF-8
\documentclass{beamer}

\usepackage{fontspec}
\usepackage{xltxtra}

%\usetheme{Boadilla}
\usetheme{PaloAlto}
%\usetheme{Berlin}
%\usecolor{}

\usepackage{minted}

%\setmainfont{Linux Libertine O}
\usepackage[english,french]{babel}

%Un peu de config de Beamer
\setbeamersize{description width=0.47cm}


\usepackage{hyperref}

\usepackage{graphicx}

\institute{Univ. de Neuchâtel}

\title{Exploiter les données}
\subtitle{Analyse statistique et stylométrie avec \textit{R}}
\author{Jean-Baptiste Camps \& Simon Gabay}
\date[FoPhil -- 15 févr. 2018]{Formation en philologie numérique:\\ encoder, exploiter, diffuser\\
12-16 février 2018}

\makeatletter 
         
        \AtBeginSection[]{% 
        \begin{frame}{Plan}%
        \small
        \tableofcontents[currentsection]%
        \end{frame} }

        %\AtBeginSubsection[]{% 
	%\begin{frame}{Plan}%
	%\small
	%\tableofcontents[currentsection,currentsubsection]%
%\end{frame} }


\makeatother 
    
    
\begin{document}

\maketitle
%\frontmatter 
  

% Présentation R



\begin{frame} 
  \frametitle{Plan} 
  \tableofcontents
\end{frame}

\section{Un peu de stylométrie…}

% Notions clés de stylométrie

% Mots fréquents

\begin{frame}{Pourquoi les mots fréquents?}

\huge

Préparez-vous à compter les \textit{l} sur la diapositive suivante…


{\normalsize Idée empruntée à Mike Kestemont!}

\end{frame}


\begin{frame}
\frametitle{}

\transduration{4}

\huge

\og{}La loi et le roi sont les meilleurs garants de la liberté civile\fg{}, dit-il.

\end{frame}

\begin{frame}{Combien de \textit{l} ?}





\end{frame}


\begin{frame}
\frametitle{5 ou 9?}

\huge

\og{}\alert{L}a \textbf{l}oi et \alert{l}e roi sont \alert{l}es mei\textbf{ll}eurs garants de \alert{l}a \textbf{l}iberté civi\textbf{l}e\fg{}, dit-i\alert{l}.

\end{frame}


\section{Statistiques descriptives}

% Moyenne… etc.



\appendix

\section{Bibliographie}

\begin{frame}[fragile]
\frametitle{Bibliographie} 

\begin{thebibliography}{}
	\bibitem[]{} 
	
\end{thebibliography}


\end{frame}


\end{document}
